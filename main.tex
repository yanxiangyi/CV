\documentclass[11pt,a4paper,sans]{moderncv}

\moderncvstyle{banking}
\moderncvcolor{black}                           
\usepackage[utf8]{inputenc}                       
\usepackage[scale=0.75]{geometry}

\usepackage{import}



\name{Xiangyi}{Yan}
\title{Curriculum Vitae}
\phone[mobile]{+86 18126237337}
\email{yanxiangyi1996@gmail.com}

\begin{document}
\makecvtitle



\section{Academic Qualifications}
\vspace{3pt}
\begin{itemize}
\item{\cventry{2015.09--2019.06 (Expected)}{Department of Computer Science and Engineering}{\large Southern University of Science and Technology (SUSTech)}{Shenzhen, China}{}{\textit{GPA: 3.85/4.0, Ranking: 3/98}}{}}
\end{itemize}
\vspace{2pt}



\section{Research Experience}
\vspace{5pt}
\begin{itemize}
\item{\cventry{2016.07--2016.08}{\textit{Humanoid Robot Lab}}{\large Tsinghua University}{Beijing, China}{Department of Mechanical Engineering}{\textbf{Knee and ankle coupled exoskeleton project.}
\newline
\vspace{2pt}Supervisor: Prof. Chenglong Fu.}}
\vspace{5pt}
\item{\cventry{2016.09--2017.05}{\textit{UAV Lab}}{\large SUSTech}{Shenzhen, China}{Department of Computer Science and Engineering}{{\textbf{Automatic UAV landing project:}}
\newline
We mainly developed quadrotor landing algorithms based on DJI SDK, combining visual recognition algorithms for precision and PID control algorithms for smoothness.
\newline
\vspace{2pt}Supervisor: Prof. Qi Hao}}
\vspace{5pt}
\item{\cventry{2018.01--2018.07}{MMLAB}{\large Chinese Academy of Sciences}{Shenzhen, China}{Shenzhen Institutes of Advanced Technology}{\textbf{Open set object detection project: }
\newline
We used deep learning algorithms to solve object detection problems while some test categories are not included in the training set.
\newline
\vspace{2pt}Supervisor: Prof. Yu Qiao.}}
\vspace{5pt}
\item{\cventry{2018.07--2018.09}{Machine Learning and Bioinformatics Lab}{\large University of California, Irvine}{Irvine, USA}{School of ICS}{\textbf{Hand pose estimation for video data project:}
\newline
To our best knowledge, we constructed the largest RGB hand pose video data set. We evaluated current state-of-the-art algorithms on it and built up several deep learning algorithms which take both structural and temporal information in to consideration.
\newline
\vspace{2pt}Supervisor: Prof. Xiaohui Xie.}}
\end{itemize}
\vspace{2pt}



\section{Internship}
\vspace{3pt}
\begin{itemize}
\item{\cventry{2018.11--now}{Medical AI Lab}{\large Tencent}{Beijing, China}{Cloud \& Smart Industries Group (CSIG)}{\textbf{Hand pose estimation for Parkinson's Disease:} 
\newline
We are currently developing and deploying deep learning algorithms to process Parkinson's disease patients' hand pose video data to help doctors to diagnose Parkinson's disease.
\newline
Supervised by Dr. Yifei Chen.}}
\end{itemize}
\vspace{2pt}
\newpage



\section{Awards and Scholarships}
\vspace{3pt}
\begin{itemize}
\item{\cventry{2016, 2017}{First class}{\large Annual Outstanding Student}{SUSTech}{\textit{Top 5\%.}}{}}
\item{\cventry{2018}{Top 0.5\%.}{\large Annual National Scholarship Mention}{SUSTech}{}{}}
\item{\cventry{2018}{Financial support for research at UC Irvine.}{\large Visiting Student Travel Grant}{SUSTech}{}{}}
\end{itemize}
\vspace{2pt}



\section{Teaching Experience (Undergraduate Helper)}
\vspace{3pt}
\begin{itemize}
\vspace{2pt}
\item{\cventry{}{Basic Program Design (Java)}{GE105}{}{\textit{Lab}}{}}
\vspace{2pt}
\item{\cventry{}{Data Structures and Algorithm Analysis}{CS203}{}{\textit{Lab}}{}}
\vspace{2pt}
\item{\cventry{}{Embedded System Microcomputer Principle}{CS301}{}{\textit{Homework}}{}}
\vspace{2pt}
\end{itemize}
\vspace{2pt}



\section{Notable Course Projects}
\vspace{3pt}
\begin{itemize}
\vspace{3pt}

\item{\cventry{}{CS302 Operating System:}{\large Pintos: Threads and User Programs}{}{}{Advanced functions implemented on a half developed operating system kernel, such as alarm clock, priority scheduling, argument passing and system calls.}}
\vspace{3pt}
\item{\cventry{}{CS303 Artificial Intelligence:}{\large Capacitied Arc Routing Problem (CARP)}{}{}{A solution for CARP based on path scanning algorithm with ellipse rule for initialization and tabu search algorithm for optimization.}}
\vspace{3pt}
\item{\cventry{}{CS309 Object Oriented Programming:}{\large Social Network Analysis for Slack (SNA4Slack)}{}{}{Social network analysis and visualization for Slack user data, such as relationship mining, hot topic mining, interaction visualization, etc.}}
\vspace{3pt}
\end{itemize}
\vspace{2pt}




\section{Professional Skills}

\vspace{6pt}

\begin{itemize}

\item {\large\textbf{Languages:}} Python, C/C++, Matlab, Java, Bash, \LaTeX.

\vspace{6pt}

\item {\large\textbf{Software:}} Xilinx ISE, ROS, SPSS, AutoCAD, Solidworks, MS Office.

\vspace{6pt}

\item {\large\textbf{Operating Systems:}} Linux, OSX, Windows.

\vspace{6pt}

\end{itemize}

\end{document}